\documentclass[a4paper, 12pt]{book}
\usepackage{amsmath, amssymb, amstext, amsthm, amssymb}
\usepackage[
    type={CC},
    modifier={by-nc-sa},
    version={4.0},
    lang={it},
]{doclicense}
\usepackage[italian]{babel}

\theoremstyle{definition}
\newtheorem{definition}[chapter]{Definizione}

\theoremstyle{remark}
\newtheorem{ex}{Esempio}

\theoremstyle{plain}
\newtheorem{theorem}{Teorema}[chapter]
\newtheorem{corollary}[theorem]{Corollario}


\title{Analisi I}
\author{Mario Merlo}
\date{\today}
\begin{document}

\maketitle

\doclicenseThis

\chapter{Insiemistica}

\begin{definition}
    Un insieme è una collezione finita o infinita di elementi
\end{definition}

\begin{theorem}
    Ciao a tutti
\end{theorem}

\end{document}