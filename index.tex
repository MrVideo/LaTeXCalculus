\documentclass[a4paper, 12pt]{book}
\usepackage{amsmath, amssymb, amstext, amsthm, amssymb}
\usepackage[
    type={CC},
    modifier={by-nc-sa},
    version={4.0},
    lang={it},
]{doclicense}
\usepackage[italian]{babel}

\theoremstyle{definition}
\newtheorem{definition}{Definizione}[chapter]

\theoremstyle{remark}
\newtheorem{ex}{Esempio}

\theoremstyle{plain}
\newtheorem{theorem}{Teorema}[chapter]
\newtheorem{corollary}{Corollario}[theorem]

\newcommand{\N}{\mathbb{N}}


\title{Analisi I}
\author{Mario Merlo}
\date{\today}
\begin{document}

\maketitle

\doclicenseThis

\chapter{Prerequisiti}

\section{Insiemistica}

\begin{definition}
    Un insieme è una collezione finita o infinita di elementi
\end{definition}

Se $A$ è un insieme ed $a$ è un suo elemento, allora si scrive: $a \in A$
Bisogna ricordare che ogni formula in matematica è \textbf{una frase con soggetto, oggetto e predicato}.

\subsection{Rappresentazione di un insieme}

Un insieme può essere rappresentato in due modi:
\begin{enumerate}
    \item \textbf{Per estensione}: ogni elemento viene enumerato in questo modo: $A = \{1, 2, 3, 4\}$
    \item \textbf{Per intensione}: l'insieme viene definito secondo una o più caratteristiche dei suoi elementi: $A = \{n \in \N \mid 1 \leq n \leq 4\}$
\end{enumerate}

\end{document}