\documentclass[a4paper, 12pt]{report}
\usepackage{amsmath, amssymb, amstext, amsthm, amssymb}
\usepackage[
    type={CC},
    modifier={by-nc-sa},
    version={4.0},
    lang={it},
]{doclicense}
\usepackage[italian]{babel}

\theoremstyle{definition}
\newtheorem{definition}{Definizione}[chapter]

\theoremstyle{remark}
\newtheorem{example}{Esempio}[definition]
\newtheorem{remark}{Osservazione}[definition]

\theoremstyle{plain}
\newtheorem{theorem}{Teorema}[chapter]
\newtheorem{corollary}{Corollario}[theorem]

\newcommand{\N}{\mathbb{N}}
\newcommand{\Z}{\mathbb{Z}}
\newcommand{\R}{\mathbb{R}}
\newcommand{\C}{\mathbb{C}}

\begin{document}

\begin{titlepage}
    \begin{center}
        \vspace*{5cm}
        \Huge{Analisi I}\\[1cm]
        \Large{Mario Merlo}\\
        \Large{https://www.github.com/MrVideo}\\
        \Large{Politecnico di Milano}\\[7,5cm]
    \end{center}
    \doclicenseThis
\end{titlepage}

\chapter{Prerequisiti}

\section{Insiemistica}

\begin{definition}
    Un insieme è una \textbf{collezione} finita o infinita di elementi.
\end{definition}

Se $A$ è un insieme ed $a$ è un suo elemento, allora si scrive: $a \in A$
Bisogna ricordare che ogni formula in matematica è \textbf{una frase con soggetto, oggetto e predicato}.

\subsection{Rappresentazione di un insieme}

Un insieme può essere rappresentato in due modi:
\begin{enumerate}
    \item \textbf{Per estensione}: ogni elemento viene enumerato in questo modo: $A = \{1, 2, 3, 4\}$
    \item \textbf{Per intensione}: l'insieme viene definito secondo una o più caratteristiche dei suoi elementi: $A = \{n \in \N \mid 1 \leq n \leq 4\}$
\end{enumerate}

\subsection{Sottoinsiemi}

\begin{definition}
    $B$ è detto \textbf{sottoinsieme} di $A$ se $\forall x \in B, x \in A$ e si scrive $B \subseteq A$.
\end{definition}

\begin{definition}
    Un insieme $A$ viene definito \textit{singleton} (o \textit{singoletto} in italiano) quando contiene un solo elemento.
\end{definition}

\begin{definition}
    Viene definito \textbf{insieme vuoto} un insieme senza alcun elemento, e si scrive $A = \emptyset$.
\end{definition}

\begin{remark}
    Comunque siano fatti gli insiemi, valgono sempre le seguenti relazioni:
    \begin{enumerate}
        \item $\emptyset \subseteq A$
        \item $A \subseteq A$
    \end{enumerate}
\end{remark}

\begin{remark}
    Si scrive $B \subset A$ per indicare che B è contenuto in A ma \textbf{non coincide} con esso: $\exists x \in A, x \notin B$. Si dice che A è \textbf{strettamente contenuto in B}.
\end{remark}

\begin{definition}
    Due insiemi sono detti \textbf{coincidenti} se $A \subseteq B \land B \subseteq A$, e si scrive $A = B$.
\end{definition}

\begin{example}
    Abbiamo i seguenti insiemi:
    \begin{itemize}
        \item $A = \{1, 2, 3, 4\}$
        \item $B = \{1, 2, 3\}$
        \item $C = \{1, 2, 5, 6\}$
        \item $D = \{n \in \N \mid 1 \leq n \leq 3\}$
    \end{itemize}
    Possiamo dire che:
    \begin{itemize}
        \item $B \subseteq A$ e anche $B \subset A$
        \item $C \nsubseteq A$ e $A \nsubseteq C$
        \item $B = D$
    \end{itemize}
\end{example}

\subsection{Operazioni tra insiemi}

Dati due insiemi $A$ e $B$, data la relazione $A, B \subseteq X$, possiamo svolgere una serie di operazioni su di essi:
\begin{enumerate}
    \item \textbf{Intersezione}: $A \cap B = \{ x \in X \mid x \in A \land x \in B\}$
    \item \textbf{Unione}: $A \cup B = \{x \in X \mid x \in A \lor x \in B\}$
    \item \textbf{Complementare}: $A^c = \overline{A} = \{ x \in X \mid x \notin A\}$
    \item \textbf{Differenza}: $A \setminus B = \{ x \in A \mid x \notin B\}$
\end{enumerate}

\subsubsection{Proprietà delle operazioni tra insiemi}
Siano $A, B, C \subseteq X$, valgono le seguenti proprietà:
\begin{enumerate}
    \item \textbf{Idempotenza}: $A \cap A = A$ e $A \cup A = A$
    \item \textbf{Commutativa}: $A \cap B = B \cap A$ e $A \cup B = B \cup A$
    \item \textbf{Associativa}: $A \cap (B \cap C) = (A \cap B) \cap C \implies A \cap B \cap C$ e $A \cup (B \cup C) = (A \cup B) \cup C \implies A \cup B \cup C$
    \item \textbf{Distributiva}: $A \cap (B \cup C) = (A \cap B) \cup (A \cap C)$ e $A \cup (B \cap C) = (A \cup B) \cap (A \cup C)$
    \item \textbf{Leggi di de Morgan}: $(A \cap B)^c = A^c \cup B^c$ e $(A \cup B)^c = A^c \cap B^c$
    \item \textbf{Doppia negazione}: $(A^c)^c = A$
    \item \textbf{Insieme vuoto}: $A \cap \emptyset = \emptyset$ e $A \cup \emptyset = A$
\end{enumerate}

\section{Relazioni e prodotto cartesiano}

\begin{definition}
    Siano $A$ e $B$ insiemi, si definisce \textbf{prodotto cartesiano} l'operazione $A \times B = \{(a, b) \mid a \in A, b \in B\}$.
\end{definition}

\begin{definition}
    Una \textbf{relazione binaria} tra due insiemi $A$ e $B$ è un sottoinsieme di $A \times B$.
\end{definition}

\begin{example}
    Siano $A, B$ due insiemi così definiti:
    \begin{itemize}
        \item $A = \{a, b, c\}$
        \item $B = \{1, 2, 3\}$
    \end{itemize}
    Allora, l'insieme $A \times B$ sarà così definito:\\
    $A \times B = \{(a, 1), (a, 2), (a, 3), (b, 1), (b, 2), (b, 3), (c, 1), (c, 2), (c, 3)\}$.\\
    Una relazione tra $A$ e $B$ è un \textbf{qualsiasi sottoinsieme} di $A \times B$:
    \begin{itemize}
        \item $A \times B$, detta \textit{relazione universale}
        \item $\emptyset$, detta \textit{relazione vuota}
        \item $R_1 = \{(a, 4), (b, 4), (c, 6)\}$, relazione generica
        \item $R_2 = \{(a, 2), (a, 4), (a, 6)\}$, relazione generica
    \end{itemize}
\end{example}

Potremmo ora porci la domanda: "\textit{Quante sono le possibili relazioni tra $A$ e $B$?}". Per rispondere, definiamo l'\textit{insieme delle parti} di un insieme.

\begin{definition}
    Sia $A$ un insieme qualunque, si chiama \textbf{insieme delle parti di $A$} l'insieme $P(A) = \{B \mid B \subseteq A\}$.
\end{definition}

\begin{example}
    Dato l'insieme $A = \{1, 2, 3\}$, il suo insieme delle parti sarà: $P(A) = \{A, \emptyset, \{1\}, \{2\}, \{3\}, \{1, 2\}, \{1, 3\}, \{2, 3\}\}$.
\end{example}

\begin{remark}
    È vero che $A \in P(A)$, ma anche $A \nsubseteq P(A)$.
\end{remark}

\begin{definition}
    Si definisce \textbf{cardinalità} di un insieme il numero di elementi di quell'insieme.
\end{definition}

\begin{remark}
    Se $A$ ha cardinalità $|A| = n$ elementi, $P(A)$ ha cardinalità $|P(A)| = 2^n$.
\end{remark}

Ora possiamo stabilire quante siano le possibili relazioni tra $A$ e $B$: \textbf{tante quanti sono i sottoinsiemi di $A \times B$}.

\begin{example}
    Consideriamo i seguenti insiemi:
    \begin{itemize}
        \item $A = \{1, 2, 3\}$
        \item $B = \{1, 2\}$
    \end{itemize}
    Possiamo certamente dire che $|A| = 3$ e $|B| = 2$, perciò sappiamo che $|A \times B| = 6$ e quindi che esistono 6 possibili relazioni tra $A$ e $B$.
\end{example}

\subsection{Proprietà delle relazioni interne}

\begin{definition}
    Si dice \textbf{relazione interna} una relazione così definita: $R \subseteq A \times A$
\end{definition}

Una relazione così definita ha le seguenti proprietà:
\begin{itemize}
    \item \textbf{Riflessiva}: se $\forall a \in A, (a, a) \in R$
    \item \textbf{Simmetrica}: se $\forall a, a' \in A$, se $(a, a') \in R$, allora $(a', a) \in R$
    \item \textbf{Antisimmetrica}: se $\forall a, a' \in A$, se $(a, a') \in R \land (a', a) \in R$, allora $a = a'$
    \item \textbf{Transitiva}: se $\forall a, a', a'' \in A$, se $(a, a') \in R \land (a', a'') \in R$, allora $(a, a'') \in R$
\end{itemize}

\begin{example}
    Dato l'insieme $A = \{1, 2, 3\}$:
    \begin{itemize}
        \item La relazione $R_1 = \{(1, 1), (2, 2)\}$ è \textit{simmetrica}, \textit{antisimmetrica}, \textit{transitiva} ma \textit{non riflessiva}.
        \item La relazione $R_2 = \{(1,1), (1,2), (2, 1), (2, 2), (3, 3)\}$ è \textit{riflessiva}, \textit{simmetrica}, \textit{transitiva} ma \textit{non antisimmetrica}.
        \item La relazione $R_3 = \{(1, 2)\}$ è \textit{antisimmetrica} e \textit{transitiva} ma \textit{non riflessiva} e \textit{non simmetrica}.
    \end{itemize}
\end{example}

\begin{example}
    Su $\N$, sia $R$: "avere la stessa parità", cioè $(n,m) \in R$ se $n$ ed $m$ sono entrambi pari o dispari.\\
    $R$ è {\it riflessiva}, {\it simmetrica}, {\it transitiva} ma {\it non antisimmetrica}.
\end{example}

\end{document}